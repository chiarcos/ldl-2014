\section{Background and Motivation}
\label{sec-background}

After
half a century of computational linguistics \citep{dostert1955georgetown},               % Georgetown experiment (1954)
quantitative typology \citep{greenberg60-quant},
empirical, corpus-based study of language \citep{francis-kucera1964},  and % Kucera & Francis (1961/63), Brown corpus
computational lexicography \citep{morris1969},
researchers in computational linguistics, natural language processing (NLP) or information technology, as well as in Digital Humanities,
 are confronted with an immense wealth of linguistic resources, that are not only growing in number, but also in their heterogeneity.
Accordingly, the limited interoperability between linguistic resources has been recognized as a major obstacle for data use and 
re-use within and across discipline boundaries, and represents one of the prime motivations for adopting Linked Data to our field. 
 
Interoperability involves two aspects \citep{ide-pustejovsky2010-interoperability}:

\smallskip
\noindent
\paragraph{(a) How to access a resource?} (Structural interoperability)
Resources use comparable formalisms to represent and to access data (formats, protocols, query languages, etc.), so that they can be accessed in a uniform way and that their information can be integrated with each other.

\smallskip
\noindent
\paragraph{(b) How to interpret information from a resource?} ({Conceptual} interoperability)
Resources share a common vocabulary, so that information from one resource can be resolved against information from another resource, e.g., grammatical descriptions can be linked to a terminology repository.

\bigskip

\noindent With the rise of the Semantic Web, new representation formalisms and novel technologies have become available, and different communities are becoming increasingly aware of the potential of these developments with respect to the challenges posited by the heterogeneity and multitude of linguistic resources available today. Many of these approaches follow the \textbf{Linked (Open) Data paradigm} \citep{bernersLee2006_linkeddata}, and this line of research, and its application to resources relevant for linguistics and/or NLP represent the focus of our work.

\subsection{Linked Data}

The Linked Open Data paradigm postulates four rules for the publication and representation of Web resources:
(1) Referred entities should be designated by using URIs,
(2) these URIs should be resolvable over HTTP,
(3) data should be represented by means of W3C standards (such as RDF),
(4) and a resource should include links to other resources.
These rules facilitate information integration, and thus, interoperability, in that they require that entities can be addressed in a globally unambiguous way (1), that they can be accessed (2) and interpreted (3), and that entities that are associated on a conceptual level are also physically associated with each other (4).

In the definition of Linked Data, the \textbf{Resource Description Framework (RDF)} receives special attention. \index{RDF (Resource Description Framework)}RDF was designed to provide metadata about resources that are available either offline (e.g., books in a library) or online (e.g., eBooks in a store). \index{RDF (Resource Description Framework)}RDF provides a generic data model based on labeled directed %(multi-)
graphs, which can be serialized in different formats. Information is expressed in terms of \emph{triples} - consisting of a \emph{property} (relation, i.e., a labeled edge) that connects a \emph{subject} (a resource, i.e., a labeled node) with its \emph{object} (another resource, or a literal, e.g., a string).
RDF resources (nodes)\footnote{
    The term `resource' is ambiguous: \emph{Linguistic} resources are structured collections of data which can be represented, for example, in RDF. In RDF, however, `resource' is the conventional name of a node in the graph, because, historically, these nodes were meant to represent objects that are described by metadata. We use the terms `node' or `concept' whenever \emph{RDF} resources are meant in ambiguous cases.
} are represented by \emph{Uniform Resource Identifiers (URIs)}. They are thus globally unambiguous in the web of data. This allows resources hosted at different locations to refer to each other, and thereby to create a network of data collections whose elements are densely interwoven.

Several database implementations for \index{RDF (Resource Description Framework)}RDF data are available, and these can be accessed using \textbf{SPARQL} \citep{prud2008sparql}, a standardized query language for \index{RDF (Resource Description Framework)}RDF data.
SPARQL uses a triple notation like RDF, except that properties and \index{RDF (Resource Description Framework)}RDF resources can be replaced by variables. SPARQL is inspired by SQL, variables can be introduced in a separate \onto{SELECT} block, and constraints on these variables are expressed as triples in the \onto{WHERE} block. SPARQL does not only support querying against individual \index{RDF (Resource Description Framework)}RDF data bases that are accessible over HTTP (`SPARQL end points'), but also, it allows us to combine information from multiple repositories (federation). \index{RDF (Resource Description Framework)}RDF can thus not only be used to \emph{establish} a network, or cloud, of data collections, but also, to \emph{query} this network directly.

Beyond its original field of application, RDF evolved into a generic format for knowledge representation. It was readily adopted by disciplines as different as biomedicine and bibliography, and eventually it became one of the building stones of the \textbf{Semantic Web}. Due to its application across discipline boundaries, \index{RDF (Resource Description Framework)}RDF is maintained by a large and active community of users and developers, and it comes with a rich infrastructure of APIs, tools, databases, query languages, and multiple sub-languages that have been developed to define data structures that are more specialized than the graphs represented by RDF. These sub-languages can be used to create \emph{reserved vocabularies} and \emph{structural constraints} for \index{RDF (Resource Description Framework)}RDF data. For example, the Web Ontology Language (OWL) defines the datatypes necessary for the representation of ontologies as an extension of RDF, i.e., \emph{classes} (concepts), \emph{instances} (individuals) and \emph{properties} (relations). 
%OWL/DL is an OWL dialect that is restricted such that the language corresponds to a description logic, i.e., a decidable fragment of first-order predicate logic. Exploiting this restriction, a number of reasoners have been developed that allow the verification of consistency constraints (\emph{axioms}) as well as methods  to draw inferences from logical relations in the ontology. If modeled as ontologies, the semantic consistency of linguistic resources can be validated and implicit information can be inferred.

The concept of Linked Data is closely coupled with the idea of \textbf{openness} (otherwise, the linking is only partially reproducible), and in 2010, the original definition of Linked Open Data has been extended with a 5 star rating system for data on the Web.\footnote{\url{http://www.w3.org/DesignIssues/LinkedData.html}, paragraph `Is your Linked Open Data 5 Star?'} The first star is achieved by publishing data on the Web (in any format) under an open license, and the second, third and fourth star require machine-readable data, a non-proprietary format, and using standards like RDF, respectively. The fifth star is achieved by linking the data to other people's data to provide context.
If (linguistic) resources are published in accordance with these rules, it is possible to follow links between existing resources to find other, related data and exploit network effects.

\subsection{Linked Data for Linguistics and NLP}
\label{sec-linked-data}

Publishing Linked Data allows resources to be globally and uniquely identified such that they can be retrieved through standard Web protocols. Moreover, resources can be easily linked to one another in a uniform fashion and thus become structurally interoperable. \citet{chiarcos-etal2012-ntrolr} identified five main benefits of Linked Data for Linguistics and NLP: 

\noindent
\paragraph{(1) Conceptual Interoperability}
Semantic Web technologies allow to provide, to maintain and to share centralized, but freely accessible terminology repositories. 
Reference to such terminology repositories facilitates conceptual interoperability as different concepts used in the annotation are backed up by externally provided definitions, and these common definitions may be employed for comparison or information integration across heterogeneous resources.

\smallskip\noindent
\paragraph{(2) Linking through URIs}
URIs provide globally unambiguous identifiers, and if resources are accessible over HTTP, it is possible to create resolvable references to URIs. Different resources developed by independent research groups can be connected into a cloud of resources.

\smallskip\noindent
\paragraph{(3) Information Integration at Query Runtime (Federation)}
Along with HTTP-accessible repositories and resolvable URIs, it is possible to combine information from physically separated  repositories in a single query at runtime: 
Resources can be uniquely identified and easily referenced from any other resource on the Web through URIs. 
Similar to hyperlinks in the HTML web, the web of data created by these links allows to navigate along these connections, 
and thereby to freely integrate information from different resources in the cloud.

\smallskip\noindent
\paragraph{(4) Dynamic Import}
When linguistic resources are interlinked by references to resolvable URIs instead of system-defined IDs (or static copies of parts from another resource), we always provide access to the most recent version of a resource. 
For community-maintained terminology repositories like the ISO TC37/SC4 Data Category Registry \citep[ISOcat]{wright2004global,windhouwer-wright2012}, for example, new categories, definitions or examples can be introduced occasionally, and this information is available immediately to anyone whose resources refer to ISOcat URIs. 
In order to preserve link consistency among Linguistic Linked Open Data resources, however, it is strongly advised to apply a proper versioning system such that backward-compatibility can be preserved: Adding concepts or examples is unproblematic, but when concepts are deleted %, renamed 
or redefined, a new version should be provided.

\smallskip\noindent
\paragraph{(5) Ecosystem}
RDF as a data exchange framework is maintained by an interdisciplinary, large and active community, and it comes with a developed infrastructure that provides APIs, database implementations, technical support and va\-li\-da\-tors for various RDF-based languages, e.g., reasoners for OWL. For developers of linguistic resources, this ecosystem can provide technological support or off-the-shelf implementations for common problems, e.g., the de\-ve\-lop\-ment of a database that is capable of support flexible, graph-based data structures as necessary for multi-layer corpora \citep{ide-suderman07-graf}.

\smallskip\noindent
\paragraph{(6) Distributed Development}
To these, \citet{chiarcos-etal-2013-ldl-intro} add that the distributed approach of the Linked Data paradigm facilitates the distributed de\-ve\-lop\-ment of a web of resources and collaboration between researchers that provide and use this data and that employ a shared set of technologies. One consequence is the emergence of interdisciplinary efforts to create large and interconnected sets of resources in linguistics and beyond. The LDL workshop series provides a forum to discuss and to facilitate such on-going developments, in particular, the emerging Linguistic Linked Open Data cloud.\footnote{
	This introduction is partially based on earlier LDL introductions. 
	It should be noted, however, that this publication provides an up-to-date description of current achievements with respect to Linked Data in Linguistics, and that it represents the primary publication documenting the progress of the Linguistic Linked Open Data cloud.
}