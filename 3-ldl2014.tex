\section{LDL-2014: The 3rd Workshop on Linked Data in Linguistics}

The goal of the 2nd Workshop on Linked Data in Linguistics (LDL-2013) has been to bring together researchers from various fields of linguistics, NLP, and information technology to present and discuss principles, case studies, and best practices for representing, publishing and linking linguistic data collections, including corpora, dictionaries, lexical networks, translation memories, thesauri, etc., infrastructures developed on that basis, their use of existing standards, and the publication and distribution policies that were adopted.

% redundant
%The workshop is continuing a series of workshops on the application of the Linked Data paradigm to linguistic data that have been initiated and organized by the Open Linguistics Working Group: The First Workshop on Linked Data in Linguistics (LDL-2012) was conducted in March 2012 at the University of Frankfurt am Main/Germany, and collocated with the 34th Annual Meeting of the German Linguistics Society (DGfS-2012). The Workshop on Multilingual Linked Open Data for Enterprises (MLODE-2012) was conducted in September 2012 at the University of Leipzig/Germany, and collocated with the 3rd Conference on Software Agents and Services for Business, Research and E-Science (SABRE-2012).

For the 2nd edition of the workshop on Linked Data in Linguistics, we invited contributions discussing the application of the Linked Open Data paradigm to linguistic data as it might provide an important step towards making linguistic data: i) easily and uniformly queryable, ii) interoperable and iii) sharable over the Web using open standards such as the HTTP protocol and the RDF data model. Recent research in this direction has lead to the emergence of a Linked Open Data cloud of linguistic resources, the Linguistic Linked Open Data (LLOD) cloud, where Linked Data principles have been applied to language resources, allowing them to be published and linked in a principled way. Although not restricted to lexical resources, these play a particularly prominent role in this context.
The topics of interest mentioned in the call for papers were the following ones:

\begin{enumerate}
\item Use cases %and project proposals 
for creation, maintenance and publication of linguistic data collections that are linked with other resources

\item Modelling linguistic data and metadata with OWL and/or RDF

\item Ontologies for linguistic data and metadata collections

\item Applications of such data, other ontologies or linked data from any subdiscipline of linguistics %(may include work in progress or project descriptions)

\item Descriptions of data sets, ideally following Linked Data principles

\item Legal and social aspects of Linguistic Linked Open Data
\end{enumerate}

CHECK ACCEPTANCE RATE
%\noindent In response to our call for papers we received 17 submissions which were all reviewed by at least two members of our program committee. On the basis of these reviews, we decided to accept 8 papers as full papers and 2 as short papers, giving an overall acceptance rate of around 50\%.

\smallskip

LDL-2014 is collocated with the 9th International Conferences for Language Resources and Evaluation (LREC-2014), and at this edition, we put a particular focus on multilingual knowledge resources and ...



