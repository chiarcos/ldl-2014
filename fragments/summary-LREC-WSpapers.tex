%&pdfLaTeX
% !TEX encoding = UTF-8 Unicode
\documentclass{article}
\usepackage{ifxetex}
\ifxetex
\usepackage{fontspec}
\setmainfont[Mapping=tex-text]{STIXGeneral}
\else
\usepackage[T1]{fontenc}
\usepackage[utf8]{inputenc}
\fi
\usepackage{textcomp}

\usepackage{amssymb}
\usepackage{fancyhdr}
\renewcommand{\headrulewidth}{0pt}
\renewcommand{\footrulewidth}{0pt}

\begin{document}

\vspace{24pt}
\textbf{LOCALIZATION and METADATA}

\vspace{12pt}
A BRIEF SURVEY OF MULTIMEDIA ANNOTATION LOCALISATION ON THE WEB OF

LINKED DATA

Gary Lefman, David Lewis and Felix Sasaki

\vspace{12pt}
FROM CLARIN COMPONENT METADATA TO LINKED OPEN DATA

Matej Durco and Menzo Windhouwer

\vspace{24pt}
\textbf{LEXICAL RESOURCES in LEMON}

\vspace{12pt}
PUBLISHING AND LINKING WORDNET USING LEMON AND RDF

John Philip McCrae, Christiane Fellbaum and Philipp Cimiano

\vspace{12pt}
USING LEMON TO MODEL LEXICAL SEMANTIC SHIFT IN DIACHRONIC LEXICAL

RESOURCES

Fahad Khan, Federico Boschetti and Francesca Frontini

\vspace{12pt}
\textbf{MORE RESOURCES INTO LOD}

\vspace{12pt}
TYPOLOGY WITH GRAPHS AND MATRICES

Steven Moran and Michael Cysouw

\vspace{12pt}
ATTACHING TRANSLATIONS TO PROPER LEXICAL SENSES IN DBNARY

Andon Tchechmedjiev, Gilles Sérasset, Jérôme Goulian and Didier

Schwab

\vspace{12pt}
TOWARDS A LINKED OPEN DATA REPRESENTATION OF A GRAMMAR TERMS INDEX

Daniel Jettka, Karim Kuropka, Cristina Vertan and Heike Zinsmeister

\vspace{12pt}
RELEASING GENRE KEYWORDS OF RUSSIAN MOVIE DESCRIPTIONS AS LINGUISTIC

LINKED OPEN DATA: AN EXPERIENCE REPORT

Andrey Kutuzov and Maxim Ionov

\vspace{12pt}
\textbf{SPECIFIC USES of Linguistic Data in LOD}

\vspace{12pt}
THE CROSS-LINGUISTIC LINKED DATA PROJECT

Robert Forkel

\vspace{12pt}
LINKING ETYMOLOGICAL DATABASES. A CASE STUDY IN GERMANIC

Christian Chiarcos and Maria Sukhareva

\vspace{12pt}
\begin{center}
\textbf{SUMMARY}
\end{center}

\vspace{24pt}
\baselineskip=12pt
\leftskip=0pt
The 10 accepted papers address a wide range of problems in the area of NLP and 
(L)LOD. These are: modeling, representation, analysis and publishing of various 
data or metadata through LOD. The most popular issues seem to be managing lexical 
databases as RDF lexicon-to-ontology standards (such as LEMON). Some papers focus 
on manipulating specific databases, such as etymological, diachronic, web, movies, 
and grammar terminology. Localization and cross-lingual issues are also considered. 
Let us present briefly the papers.

The paper A BRIEF SURVEY OF MULTIMEDIA ANNOTATION LOCALISATION ON THE WEB OF LINKED 
DATA (Lefman, Lewis and Sasaki) investigates the localization of multimedia ontologies 
and Linked Data frameworks focusing on Flickr. The authors view Linguistic Linked 
Data as a mediator between multimedia annotation in social media and the Web of 
Linked Data. The paper FROM CLARIN COMPONENT METADATA TO LINKED OPEN DATA (Durèo 
and Windhouwer) presents in detail the ways of relating CMDI resource descriptions 
into LOD. Thus, the metadata is RDF-ized and visible via SPARQL endpoints. The 
paper PUBLISHING AND LINKING WORDNET USING LEMON AND RDF (McCrae, Fellbaum and 
Cimiano) proposes a strategy for publishing Princeton Wordnet as linked data through 
an open model. The advantage of this approach is that it provides linking also 
to the resources which have been already integrated into Wordnet. The paper USING 
LEMON TO MODEL LEXICAL SEMANTIC SHIFT IN DIACHRONIC LEXICAL RESOURCES (Khan, Boschetti 
and Frontini) proposes an ontology-based extension of the LEMON model (called LEMON-DIA) 
for representing the lexical semantic change in temporal context. The focus is 
on the implementation of philosophical properties, such as `perdurant'. The paper 
TYPOLOGY WITH GRAPHS AND MATRICES concentrates on the three-fold representation 
(graphs, tables, matrices) of the same data source. Linguistic databases are accessed 
via Linked Data. The paper ATTACHING TRANSLATIONS TO PROPER LEXICAL SENSES IN DBNARY 
(Tchechmedjiev, Sérasset, Goulian and Schwab) presents the DBNARY project, which 
aims at extracting LOD from Wiktionaries of various languages. More specifically, 
the authors present a similarity technique for disambiguation of linked translations. 
The paper TOWARDS A LINKED OPEN DATA REPRESENTATION OF A GRAMMAR TERMS INDEX (Jettka, 
Kuropka, Vertan and Zinsmeister) introduces an ongoing work on a Linked Open Data 
set of German grammar terms, called HyperGramm. The proposed strategy is applicable 
also to other languages with the availability of similar resources. The paper RELEASING 
GENRE KEYWORDS OF RUSSIAN MOVIE DESCRIPTIONS AS LINGUISTIC LINKED OPEN DATA: AN 
EXPERIENCE REPORT (Kutuzov and Ionov) describes a work on publishing genre-classified 
movie keywords as LOD using the LEMON model. The resource is also linked to Russian 
DBPedia Wiktionary. The paper THE CROSS LINGUISTIC LINKED DATA PROJECT (Forkel) 
introduces a data model within an ongoing initiative on establishing a platform 
for interoperability among various language resources. In this context LOD plays 
the important role of publishing strategy for the datasets. The paper LINKING ETYMOLOGICAL 
DATABASES. A CASE STUDY IN GERMANIC (Chiarcos and Sukhareva) focuses on first attempt 
of modeling German etymological datasets in LOD. The work is challenging, since 
it handles different language stages. LEMON was implemented as a basic standard, 
but it will be further developed into LEMON-conformant one in order to meet the 
diachronic data requirements.

\newpage

\end{document}
