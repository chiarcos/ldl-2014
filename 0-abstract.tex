% 150-200 words
\abstract{
	The Linked Data in Linguistics (LDL) workshop series brings together researchers from various fields of linguistics, natural language processing, and information technology to present and discuss principles, case studies, and best practices for representing, publishing and linking linguistic data collections. A  major outcome of our work is the Linguistic Linked Open Data (LLOD) cloud, an LOD (sub-)cloud of linguistic resources, which covers various linguistic data bases, lexicons, corpora, terminology and metadata repositories.\\
%	
	As a general introduction into the topic, we describe the concept of Linked Data, its application in linguistics and the development of the Linguistic Linked Open Data (LLOD) cloud since LDL-2013. We present the contributions of LDL-2014, the associated data challenge and its results and present the newly compiled LLOD cloud diagram. \\
%	
	The third instantiation of this series, collocated with the 9th Language Resources and Evaluation Conference (LREC-2014), May 27th, 2014, in Reykjavik, Iceland, is specifically dedicated to the study of Multilingual Knowledge Resources and Natural Language Processing, although contributions with respect to any application of Linked Data to linguistically and/or NLP-relevant resources are welcome.
\\ 
\newline 
\Keywords{Linked Data in Linguistics (LDL), Linguistic Linked Open Data (LLOD) cloud}}
