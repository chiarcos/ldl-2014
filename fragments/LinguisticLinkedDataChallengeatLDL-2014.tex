%&pdfLaTeX
% !TEX encoding = UTF-8 Unicode
\documentclass{article}
\usepackage{ifxetex}
\ifxetex
\usepackage{fontspec}
\setmainfont[Mapping=tex-text]{STIXGeneral}
\else
\usepackage[T1]{fontenc}
\usepackage[utf8]{inputenc}
\fi
\usepackage{textcomp}

\usepackage{amssymb}
\usepackage{fancyhdr}
\renewcommand{\headrulewidth}{0pt}
\renewcommand{\footrulewidth}{0pt}

\begin{document}

\baselineskip=13pt
In addition to the main workshop, we introduced an open challenge for the creation 
of datasets for linguistics according to linked data principles. This challenge 
required submissions of new linked datasets and was evaluated by reviewers on a 
number of technical grounds as follows: 

\vspace{13pt}
\leftskip=36pt
\parindent=-18pt
{\large{}1. }Availability

\leftskip=72pt
{\large{}○ }Use of Linked Data and RDF.

{\large{}○ }Hosted on a publicly accessible server and be available both during 
the period of the evaluation and beyond.

{\large{}○ }Use of an open license.

\leftskip=36pt
{\large{}2. }Quality of Resource

\leftskip=72pt
{\large{}○ }Represents useful linguistically or NLP-relevant information.

{\large{}○ }Reuses relevant standards and models.

{\large{}○ }Contains complex, non-trivial information, e.g., multiple levels 
of annotation.

\leftskip=36pt
{\large{}3. }Linking

\leftskip=72pt
{\large{}○ }Links to external resources.

{\large{}○ }Reuse of existing properties and categories.

\leftskip=36pt
{\large{}4. }Impact/usefulness of the resource

\leftskip=72pt
{\large{}○ }Relevant and likely to be reused by many researchers in NLP and wider 
fields.

{\large{}○ }Uses linked data to improve the quality of and access to the resource.

\leftskip=36pt
{\large{}5. }Originality

\leftskip=72pt
{\large{}○ }Represents a type of resource or a community currently underrepresented 
in (L)LOD cloud activities

{\large{}○ }Facilitates novel and unforeseen applications or use cases (as described 
by the authors) enabled through Linked Data technology.

\vspace{13pt}
\leftskip=0pt
\parindent=0pt
This year there were five accepted submissions to the challenge and from those 
we chose two joint winners and one highly commended paper. The winners this year 
were ``DBnary: Wiktionary as Linked Data for 12 Language Editions with Enhanced 
Translation Relations'' by Gilles Sérraset and Andon Tchechmedjiev and ``Linked-data 
based domain-specific sentiment lexicons'' by Gabriela Vulcu, Raul Lario Monje, 
Mario Munoz, Paul Buitelaar and Carlos A. Iglesias, describing the EuroSentiment 
lexicon. The DBnary work describes the extraction of Multilingual data from Wiktionary 
based on 12 language editions of Wiktionary, and as such represents a large and 
important lexical resource that should have application in many linguistic areas. 
The winner described the creation of a lexicon for the EuroSentiment project, which 
tackles the important field of sentiment analysis through the use of sophisticated 
linguistic processing. The resource described extends the \textit{lemon} model 
with the MARL vocabulary to provide a lexicon that is unique in the field of sentiment 
analysis due to its linguistic sophistication. Finally, we also highly commend 
the work presented in ``A multilingual semantic network as linked data: Lemon-BabelNet'' 
by Maud Ehrmann, Francesco Cecconi, Daniele Vannelle, John P. McCrae, Philipp Cimiano 
and Roberto Navigli, which describes the expression of BabelNet using the \textit{lemon} 
vocabulary. BabelNet is one of the largest lexical resources created to date and 
its linked data version at over 1 billion triples will be one of the largest resources 
in the LLOD cloud. As such, the clear usefulness of the resource as a target for 
linking and also the use of the widely-used \textit{lemon} model make this conversion 
a highly valuable resource for the community as noted by the reviewers. Finally, 
we will note that our two runner-up participants ``PDEV-LEMON: A linked data implementation 
of the pattern dictionary of English verbs based on the \textit{lemon} model'' 
by Ismail El Maarouf et al., and ``Linked Hypernyms Dataset - Generation Framework 
and Use Cases'' by TomÁš Kliegr et al. were also well received as resources that 
continue to grow the linguistic linked open data cloud and are likely to find application 
for a number of works in linguistics and natural language processing.

\newpage

\end{document}
