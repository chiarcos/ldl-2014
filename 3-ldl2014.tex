\section{LDL-2014: The 3rd Workshop on Linked Data in Linguistics}

The goal of the 2nd Workshop on Linked Data in Linguistics (LDL-2013) has been to bring together researchers from various fields of linguistics, NLP, and information technology to present and discuss principles, case studies, and best practices for representing, publishing and linking linguistic data collections, including corpora, dictionaries, lexical networks, translation memories, thesauri, etc., infrastructures developed on that basis, their use of existing standards, and the publication and distribution policies that were adopted.

% redundant
%The workshop is continuing a series of workshops on the application of the Linked Data paradigm to linguistic data that have been initiated and organized by the Open Linguistics Working Group: The First Workshop on Linked Data in Linguistics (LDL-2012) was conducted in March 2012 at the University of Frankfurt am Main/Germany, and collocated with the 34th Annual Meeting of the German Linguistics Society (DGfS-2012). The Workshop on Multilingual Linked Open Data for Enterprises (MLODE-2012) was conducted in September 2012 at the University of Leipzig/Germany, and collocated with the 3rd Conference on Software Agents and Services for Business, Research and E-Science (SABRE-2012).

For the 2nd edition of the workshop on Linked Data in Linguistics, we invited contributions discussing the application of the Linked Open Data paradigm to linguistic data as it might provide an important step towards making linguistic data: i) easily and uniformly queryable, ii) interoperable and iii) sharable over the Web using open standards such as the HTTP protocol and the RDF data model. Recent research in this direction has lead to the emergence of a Linked Open Data cloud of linguistic resources, the Linguistic Linked Open Data (LLOD) cloud, where Linked Data principles have been applied to language resources, allowing them to be published and linked in a principled way. Although not restricted to lexical resources, these play a particularly prominent role in this context.
The topics of interest mentioned in the call for papers were the following ones:

\begin{enumerate}
\item Use cases %and project proposals 
for creation, maintenance and publication of linguistic data collections that are linked with other resources

\item Modelling linguistic data and metadata with OWL and/or RDF

\item Ontologies for linguistic data and metadata collections

\item Applications of such data, other ontologies or linked data from any subdiscipline of linguistics %(may include work in progress or project descriptions)

\item Descriptions of data sets, ideally following Linked Data principles

\item Legal and social aspects of Linguistic Linked Open Data
\end{enumerate}

CHECK ACCEPTANCE RATE
%\noindent In response to our call for papers we received 17 submissions which were all reviewed by at least two members of our program committee. On the basis of these reviews, we decided to accept 8 papers as full papers and 2 as short papers, giving an overall acceptance rate of around 50\%.

\smallskip

LDL-2014 is collocated with the 9th International Conferences for Language Resources and Evaluation (LREC-2014), and at this edition, we put a particular focus on multilingual knowledge resources and ...

TODO: reorganize in accordance with the datatype categories in the diagram ?

The 10 accepted papers address a wide range of problems in the area of NLP and 
(L)LOD. These are: modeling, representation, analysis and publishing of various 
data or metadata through LOD. The most popular issues seem to be managing lexical 
databases as RDF lexicon-to-ontology standards (such as LEMON). Some papers focus 
on manipulating specific databases, such as etymological, diachronic, web, movies, 
and grammar terminology. Localization and cross-lingual issues are also considered. 
Let us present briefly the papers.

The paper A BRIEF SURVEY OF MULTIMEDIA ANNOTATION LOCALISATION ON THE WEB OF LINKED 
DATA (Lefman, Lewis and Sasaki) investigates the localization of multimedia ontologies 
and Linked Data frameworks focusing on Flickr. The authors view Linguistic Linked 
Data as a mediator between multimedia annotation in social media and the Web of 
Linked Data. The paper FROM CLARIN COMPONENT METADATA TO LINKED OPEN DATA (Durèo 
and Windhouwer) presents in detail the ways of relating CMDI resource descriptions 
into LOD. Thus, the metadata is RDF-ized and visible via SPARQL endpoints. The 
paper PUBLISHING AND LINKING WORDNET USING LEMON AND RDF (McCrae, Fellbaum and 
Cimiano) proposes a strategy for publishing Princeton Wordnet as linked data through 
an open model. The advantage of this approach is that it provides linking also 
to the resources which have been already integrated into Wordnet. The paper USING 
LEMON TO MODEL LEXICAL SEMANTIC SHIFT IN DIACHRONIC LEXICAL RESOURCES (Khan, Boschetti 
and Frontini) proposes an ontology-based extension of the LEMON model (called LEMON-DIA) 
for representing the lexical semantic change in temporal context. The focus is 
on the implementation of philosophical properties, such as `perdurant'. The paper 
TYPOLOGY WITH GRAPHS AND MATRICES concentrates on the three-fold representation 
(graphs, tables, matrices) of the same data source. Linguistic databases are accessed 
via Linked Data. The paper ATTACHING TRANSLATIONS TO PROPER LEXICAL SENSES IN DBNARY 
(Tchechmedjiev, Sérasset, Goulian and Schwab) presents the DBNARY project, which 
aims at extracting LOD from Wiktionaries of various languages. More specifically, 
the authors present a similarity technique for disambiguation of linked translations. 
The paper TOWARDS A LINKED OPEN DATA REPRESENTATION OF A GRAMMAR TERMS INDEX (Jettka, 
Kuropka, Vertan and Zinsmeister) introduces an ongoing work on a Linked Open Data 
set of German grammar terms, called HyperGramm. The proposed strategy is applicable 
also to other languages with the availability of similar resources. The paper RELEASING 
GENRE KEYWORDS OF RUSSIAN MOVIE DESCRIPTIONS AS LINGUISTIC LINKED OPEN DATA: AN 
EXPERIENCE REPORT (Kutuzov and Ionov) describes a work on publishing genre-classified 
movie keywords as LOD using the LEMON model. The resource is also linked to Russian 
DBPedia Wiktionary. The paper THE CROSS LINGUISTIC LINKED DATA PROJECT (Forkel) 
introduces a data model within an ongoing initiative on establishing a platform 
for interoperability among various language resources. In this context LOD plays 
the important role of publishing strategy for the datasets. The paper LINKING ETYMOLOGICAL 
DATABASES. A CASE STUDY IN GERMANIC (Chiarcos and Sukhareva) focuses on first attempt 
of modeling German etymological datasets in LOD. The work is challenging, since 
it handles different language stages. LEMON was implemented as a basic standard, 
but it will be further developed into LEMON-conformant one in order to meet the 
diachronic data requirements.




\subsection{Data challenge}

In addition to the main workshop, we introduced an open challenge for the creation of datasets for linguistics according to linked data principles. This challenge required submissions of new linked datasets and was evaluated by reviewers on a number of technical grounds as follows: 

\baselineskip=13pt
In addition to the main workshop, we introduced an open challenge for the creation 
of datasets for linguistics according to linked data principles. This challenge 
required submissions of new linked datasets and was evaluated by reviewers on a 
number of technical grounds as follows: 

\vspace{13pt}
\leftskip=36pt
\parindent=-18pt
{\large{}1. }Availability

\leftskip=72pt
{\large{}? }Use of Linked Data and RDF.

{\large{}? }Hosted on a publicly accessible server and be available both during 
the period of the evaluation and beyond.

{\large{}? }Use of an open license.

\leftskip=36pt
{\large{}2. }Quality of Resource

\leftskip=72pt
{\large{}? }Represents useful linguistically or NLP-relevant information.

{\large{}? }Reuses relevant standards and models.

{\large{}? }Contains complex, non-trivial information, e.g., multiple levels 
of annotation.

\leftskip=36pt
{\large{}3. }Linking

\leftskip=72pt
{\large{}? }Links to external resources.

{\large{}? }Reuse of existing properties and categories.

\leftskip=36pt
{\large{}4. }Impact/usefulness of the resource

\leftskip=72pt
{\large{}? }Relevant and likely to be reused by many researchers in NLP and wider 
fields.

{\large{}? }Uses linked data to improve the quality of and access to the resource.

\leftskip=36pt
{\large{}5. }Originality

\leftskip=72pt
{\large{}? }Represents a type of resource or a community currently underrepresented 
in (L)LOD cloud activities

{\large{}? }Facilitates novel and unforeseen applications or use cases (as described 
by the authors) enabled through Linked Data technology.

\vspace{13pt}
\leftskip=0pt
\parindent=0pt
This year there were five accepted submissions to the challenge and from those 
we chose two joint winners and one highly commended paper. The winners this year 
were ``DBnary: Wiktionary as Linked Data for 12 Language Editions with Enhanced 
Translation Relations'' by Gilles Sérraset and Andon Tchechmedjiev and ``Linked-data 
based domain-specific sentiment lexicons'' by Gabriela Vulcu, Raul Lario Monje, 
Mario Munoz, Paul Buitelaar and Carlos A. Iglesias, describing the EuroSentiment 
lexicon. The DBnary work describes the extraction of Multilingual data from Wiktionary 
based on 12 language editions of Wiktionary, and as such represents a large and 
important lexical resource that should have application in many linguistic areas. 
The winner described the creation of a lexicon for the EuroSentiment project, which 
tackles the important field of sentiment analysis through the use of sophisticated 
linguistic processing. The resource described extends the \textit{lemon} model 
with the MARL vocabulary to provide a lexicon that is unique in the field of sentiment 
analysis due to its linguistic sophistication. Finally, we also highly commend 
the work presented in ``A multilingual semantic network as linked data: Lemon-BabelNet'' 
by Maud Ehrmann, Francesco Cecconi, Daniele Vannelle, John P. McCrae, Philipp Cimiano 
and Roberto Navigli, which describes the expression of BabelNet using the \textit{lemon} 
vocabulary. BabelNet is one of the largest lexical resources created to date and 
its linked data version at over 1 billion triples will be one of the largest resources 
in the LLOD cloud. As such, the clear usefulness of the resource as a target for 
linking and also the use of the widely-used \textit{lemon} model make this conversion 
a highly valuable resource for the community as noted by the reviewers. Finally, 
we will note that our two runner-up participants ``PDEV-LEMON: A linked data implementation 
of the pattern dictionary of English verbs based on the \textit{lemon} model'' 
by Ismail El Maarouf et al., and ``Linked Hypernyms Dataset - Generation Framework 
and Use Cases'' by TomÁš Kliegr et al. were also well received as resources that 
continue to grow the linguistic linked open data cloud and are likely to find application 
for a number of works in linguistics and natural language processing.
