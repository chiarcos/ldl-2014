\begin{abstract}
	The Linked Data in Linguistics (LDL) workshop series brings together researchers from various fields of linguistics, natural language processing, and information technology to present and discuss principles, case studies, and best practices for representing, publishing and linking linguistic data collections. A  major outcome of our work is the Linguistic Linked Open Data (LLOD) cloud, an LOD (sub-)cloud of linguistic resources, which covers various linguistic data bases, lexicons, corpora, terminology and metadata repositories.\\
%	
	As a general introduction into the topic, we describe the concept of Linked Data, its application in linguistics and the Linguistic Linked Open Data (LLOD) cloud, its history, and in particular, community activities since the last LDL workshop at LREC-2014. In addition, we present the contributions of LDL-2015 as well as the newly compiled LLOD cloud diagram. \\
%	
	The fourth instantiation of this series, collocated with ACL-IJCNLP 2015, the 53rd Annual Meeting of the Association of Computational Linguistics and the 7th Joint Conference on Natural Language Processing of the Asian Federation of Natural Language Processing, July 31st, 2015, in Beijing, China, is specifically dedicated to Resources and Applications. With the rapid development of the LLOD cloud in last years, and the recent surge of interest in Linked Data in (Computational) Linguistics, the focus of our community is increasingly shifting towards applications of this data, especially -- but not limited to -- Natural Language Processing. 
	Nevertheless, contributions with respect to any application of Linked Data to linguistically and/or NLP-relevant resources are welcome, as well.

%% not in ACL-2015 style
% \Keywords{Linked Data in Linguistics (LDL), Linguistic Linked Open Data (LLOD) cloud}

\end{abstract}